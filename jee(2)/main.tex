\let\negmedspace\undefined
\let\negthickspace\undefined
\documentclass[journal]{IEEEtran}
\usepackage[a5paper, margin=10mm, onecolumn]{geometry}
%\usepackage{lmodern} % Ensure lmodern is loaded for pdflatex
\usepackage{tfrupee} % Include tfrupee package

\setlength{\headheight}{1cm} % Set the height of the header box
\setlength{\headsep}{0mm}     % Set the distance between the header box and the top of the text

\usepackage{gvv-book}
\usepackage{gvv}
\usepackage{cite}
\usepackage{amsmath,amssymb,amsfonts,amsthm}
\usepackage{algorithmic}
\usepackage{graphicx}
\usepackage{textcomp}
\usepackage{xcolor}
\usepackage{txfonts}
\usepackage{listings}
\usepackage{enumitem}
\usepackage{mathtools}
\usepackage{gensymb}
\usepackage{comment}
\usepackage[breaklinks=true]{hyperref}
\usepackage{tkz-euclide} 
\usepackage{listings}
% \usepackage{gvv}                                        
\def\inputGnumericTable{}                                 
\usepackage[latin1]{inputenc}                                
\usepackage{color}                                            
\usepackage{array}                                            
\usepackage{longtable}                                       
\usepackage{calc}                                             
\usepackage{multirow}                                         
\usepackage{hhline}                                           
\usepackage{ifthen}                                           
\usepackage{lscape}


\renewcommand{\thefigure}{\theenumi}
\renewcommand{\thetable}{\theenumi}
\setlength{\intextsep}{10pt} % Space between text and floats


\numberwithin{equation}{enumi}
\numberwithin{figure}{enumi}
\renewcommand{\thetable}{\theenumi}


% Marks the beginning of the document
\begin{document}
\bibliographystyle{IEEEtran}
\vspace{3cm}

\title{jee-main-maths-05-09-2020-shift-1}
\author{EE24BTECH11030 - J.KEDARANANDA}
% \maketitle
% \newpage
% \bigskip
{\let\newpage\relax\maketitle}
\renewcommand{\thefigure}{\theenumi}
\renewcommand{\thetable}{\theenumi}
\begin{enumerate}
    \item If the four complex numbers z, $\bar{z}$ , $\bar{z}$ - 2Re($\bar{z}$)  and z - 2Re(z) represent the vertices of a square of side 4 units in the Argand plane, then $|z|$ is equal to: \\ 
    \begin{multicols}{4}
    \begin{enumerate}
        \item 2
        \item 4
        \item 4$\sqrt{2}$
        \item 2$\sqrt{2}$
    \end{enumerate}
    \end{multicols}
    \item If $\int(e^{2x} + 2e^{x} - e^{-x} - 1)e^{(e^x + e^{-x})}$ dx = g(x)$e^{(e^x + e^{-x})}$ + c , where c is a constant of integration,then g(0) is equal to : \\
    \begin{multicols}{4}
    \begin{enumerate}
        \item 2
        \item e
        \item 1
        \item $e^2$
    \end{enumerate} 
    \end{multicols}
    \item The negation of the Boolean expression x $\leftrightarrow$ $\sim$ y  is equivalent to : \\

    \begin{enumerate}
        \item $(x \land y) \land (\sim x \lor \sim y)$
        \item $(x \land y) \lor (\sim x \land \sim y)$
        \item $(x \land \sim y) \lor (\sim x \land  y)$
        \item $(\sim x \land y) \lor (\sim x \land \sim y)$
    \end{enumerate}
    \item If $\alpha$ is positive root of the equation, p(x) =$x^2-x-2=0$, then $\lim_{x \to \alpha^+} \frac{\sqrt{1 - \cos{(p(x))}}}{x + \alpha - 4}$ is equal to: \\
    \begin{multicols}{4}
    \begin{enumerate}
        \item $\frac{1}{2}$
        \item $\frac{3}{\sqrt{2}}$
        \item $\frac{3}{2}$
        \item $\frac{1}{\sqrt{2}}$
    \end{enumerate} 
    \end{multicols}
    \item If the co-ordinates of two points $\vec{A}$ and $\vec{B}$ are $\brak{\sqrt{7},0}$ and $\brak{-\sqrt{7},0}$respectively and $\vec{P}$ is any point on the conic, $9x^2+16y^2=144$, then PA+PB is equal to : \\
    \begin{multicols}{4}
    \begin{enumerate}
        \item $6$
        \item $16$
        \item $9$
        \item $8$
    \end{enumerate} 
    \end{multicols}
    \item The natural number m, for which the coefficient of x in the binomial expansion of ${\left(x^m + \frac{1}{x^2}\right)}^{22}$ is 1540 , is $\cdots$ \\
    \item Four fair dice are thrown independently 27 times. Then the expected number of times,at least two dice show up a three or a five, is $\cdots$ \\
    \item Let f(x) = x. $\left[\frac{x}{2}\right]$ , for -10 $<$ x $<$ 10, where [t] denotes the greatest integer function. Then the number of points of discontinuity of f is equal to $\cdots$ \\
    \item The number of words, with or without meaning, that can be formed by taking 4 letters at a time from the letters of the word 'SYLLABUS' such that two letters are distinct and two letters are alike, is \\
    \item If the line, $2x-y+3=0$ is at a distance $\frac{1}{\sqrt{5}}$ and $\frac{2}{\sqrt{5}}$ from the lines $4x - 2y + \alpha =0$ and $6x - 3y + \beta =0$, respectively, then the sum of all possible values of $\alpha$ and $\beta$ is
\end{enumerate}
\end{document}

