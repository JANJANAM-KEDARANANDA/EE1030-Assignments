\let\negmedspace\undefined
\let\negthickspace\undefined
\documentclass[journal]{IEEEtran}
\usepackage[a5paper, margin=10mm, onecolumn]{geometry}
%\usepackage{lmodern} % Ensure lmodern is loaded for pdflatex
\usepackage{tfrupee} % Include tfrupee package

\setlength{\headheight}{1cm} % Set the height of the header box
\setlength{\headsep}{0mm}     % Set the distance between the header box and the top of the text

\usepackage{gvv-book}
\usepackage{gvv}
\usepackage{cite}
\usepackage{amsmath,amssymb,amsfonts,amsthm}
\usepackage{algorithmic}
\usepackage{graphicx}
\usepackage{textcomp}
\usepackage{xcolor}
\usepackage{txfonts}
\usepackage{listings}
\usepackage{enumitem}
\usepackage{mathtools}
\usepackage{gensymb}
\usepackage{comment}
\usepackage[breaklinks=true]{hyperref}
\usepackage{tkz-euclide} 
\usepackage{listings}
% \usepackage{gvv}                                        
\def\inputGnumericTable{}                                 
\usepackage[latin1]{inputenc}                                
\usepackage{color}                                            
\usepackage{array}                                            
\usepackage{longtable}                                       
\usepackage{calc}                                             
\usepackage{multirow}                                         
\usepackage{hhline}                                           
\usepackage{ifthen}                                           
\usepackage{lscape}
\usepackage{circuitikz}


\renewcommand{\thefigure}{\theenumi}
\renewcommand{\thetable}{\theenumi}
\setlength{\intextsep}{10pt} % Space between text and floats


\numberwithin{equation}{enumi}
\numberwithin{figure}{enumi}
\renewcommand{\thetable}{\theenumi}


% Marks the beginning of the document
\begin{document}
\bibliographystyle{IEEEtran}
\vspace{3cm}

\title{GATE 2013 MA(14-26)}
\author{EE24BTECH11030 - J.KEDARANANDA}
% \maketitle
% \newpage
% \bigskip
{\let\newpage\relax\maketitle}
\renewcommand{\thefigure}{\theenumi}
\renewcommand{\thetable}{\theenumi}
\begin{enumerate}

\item Let $c \in \mathbb{Z}_3$ be such that $\frac{\mathbb{Z}_3[X]}{(X^3 + cX + X + 1)}$ is a field. Then $c$ is equal to \underline{\hspace{1cm}}.
\bigskip

\item Let $V = C^1[0,1]$, $X = C([0,1], \|\;\|_{\infty})$ and $Y = C([0,1], \|\;\|_2)$. Then $V$ is

\begin{multicols}{2}
    \begin{enumerate}
        \item dense in $X$ but NOT in $Y$
        \item dense in $Y$ but NOT in $X$
        \item dense in both $X$ and $Y$
        \item neither dense in $X$ nor dense in $Y$
    \end{enumerate}
\end{multicols}
\bigskip

\item Let $T : (C([0,1], \|\;\|_{\infty}) \to \mathbb{R}$ be defined by $T(f) = \int_0^1 2xf(x) \, dx$ for all $f \in C([0,1])$. Then $\|T\|$ is equal to \underline{\hspace{1cm}}.
\bigskip

\item Let $\tau_1$ be the usual topology on $\mathbb{R}$. Let $\tau_2$ be the topology on $\mathbb{R}$ generated by $\mathcal{B} = \{(a, b) \subset \mathbb{R} : -\infty < a < b < \infty\}$. Then the set $\{ x \in \mathbb{R} : 4 \sin^2 x \leq 1 \} \cup \left\{ \frac{\pi}{2} \right\}$ is

    \begin{enumerate}
        \item closed in $(\mathbb{R}, \tau_1)$ but NOT in $(\mathbb{R}, \tau_2)$
        \item closed in $(\mathbb{R}, \tau_2)$ but NOT in $(\mathbb{R}, \tau_1)$
        \item closed in both $(\mathbb{R}, \tau_1)$ and $(\mathbb{R}, \tau_2)$
        \item neither closed in $(\mathbb{R}, \tau_1)$ nor closed in $(\mathbb{R}, \tau_2)$
    \end{enumerate}
\bigskip

\item Let $X$ be a connected topological space such that there exists a non-constant continuous function $f : X \to \mathbb{R}$, where $\mathbb{R}$ is equipped with the usual topology. Let $f(X) = \{f(x) : x \in X\}$. Then
    \begin{enumerate}
        \item $X$ is countable but $f(X)$ is uncountable
        \item $f(X)$ is countable but $X$ is uncountable
        \item both $f(X)$ and $X$ are countable
        \item both $f(X)$ and $X$ are uncountable
    \end{enumerate}
\bigskip

\item Let $d_1$ and $d_2$ denote the usual metric and the discrete metric on $\mathbb{R}$, respectively. Let $f : (\mathbb{R}, d_1) \to (\mathbb{R}, d_2)$ be defined by $f(x) = x, x \in \mathbb{R}$. Then

    \begin{enumerate}
        \item $f$ is continuous but $f^{-1}$ is NOT continuous
        \item $f^{-1}$ is continuous but $f$ is NOT continuous
        \item both $f$ and $f^{-1}$ are continuous
        \item neither $f$ nor $f^{-1}$ is continuous
    \end{enumerate}
\bigskip

\item If the trapezoidal rule with single interval $[0,1]$ is exact for approximating the integral $\int_0^1 (x^3 - c x^2) dx$, then the value of $c$ is equal to \underline{\hspace{1cm}}.
\bigskip

\item Suppose that the Newton-Raphson method is applied to the equation $2x^2 + 1 - e^{x^2} = 0$ with an initial approximation $x_0$ sufficiently close to zero. Then, for the root $x = 0$, the order of convergence of the method is equal to \underline{\hspace{1cm}}.
\bigskip
    \item The minimum possible order of a homogeneous linear ordinary differential equation with real constant coefficients having $x^2 \sin(x)$ as a solution is equal to \underline{\hspace{1cm}}.
    
    \bigskip

    \item The Lagrangian of a system in terms of polar coordinates $(r, \theta)$ is given by
    \[
    L = \frac{1}{2} m \dot{r}^2 + \frac{1}{2} m \left( r^2 + r^2 \dot{\theta}^2 \right) - m g r \left( 1 - \cos(\theta) \right),
    \]
    where $m$ is the mass, $g$ is the acceleration due to gravity, and $\dot{s}$ denotes the derivative of $s$ with respect to time. Then the equations of motion are
    \begin{enumerate}
        \item $ \quad 2 \ddot{r} = r \dot{\theta}^2 - g \left( 1 - \cos(\theta) \right), \quad \frac{d}{dt} \left( r^2 \dot{\theta} \right) = - g r \sin(\theta)$
        \item $ \quad 2 \ddot{r} = r \dot{\theta}^2 + g \left( 1 - \cos(\theta) \right), \quad \frac{d}{dt} \left( r^2 \dot{\theta} \right) = - g r \sin(\theta)$
        \item $ \quad 2 \ddot{r} = r \dot{\theta}^2 - g \left( 1 - \cos(\theta) \right), \quad \frac{d}{dt} \left( r^2 \dot{\theta} \right) = g r \sin(\theta)$
        \item $ \quad 2 \ddot{r} = r \dot{\theta}^2 + g \left( 1 - \cos(\theta) \right), \quad \frac{d}{dt} \left( r^2 \dot{\theta} \right) = g r \sin(\theta)$
    \end{enumerate}

    \bigskip

    \item If $y(x)$ satisfies the initial value problem 
    \[
    (x^2 + y) dx = x \, dy, \quad y(1) = 2,
    \]
    then $y(2)$ is equal to \underline{\hspace{1cm}}.
    
    \bigskip

    \item It is known that Bessel functions $J_n(x)$, for $n \geq 0$, satisfy the identity
    \[
    e^{z \left( t - \frac{1}{t} \right)} = J_0(x) + \sum_{n=1}^\infty J_n(x) \left( t^n + \frac{(-1)^n}{t^n} \right)
    \]
    for all $t > 0$ and $x \in \mathbb{R}$. The value of $J_0 \left( \frac{\pi}{3} \right) + 2 \sum_{n=1}^\infty J_{2n} \left( \frac{\pi}{3} \right)$ is equal to \underline{\hspace{1cm}}.
    
    \bigskip

    \item Let $X$ and $Y$ be two random variables having the joint probability density function
    \[
    f(x,y) = \begin{cases}
        2 & \text{if } 0 < x < y < 1 \\
        0 & \text{otherwise}
    \end{cases}
    \]
    Then the conditional probability $P ( X \leq \frac{2}{3}|$

\end{enumerate}
\end{document}





