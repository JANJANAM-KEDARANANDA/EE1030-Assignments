\let\negmedspace\undefined
\let\negthickspace\undefined
\documentclass[journal]{IEEEtran}
\usepackage[a5paper, margin=10mm, onecolumn]{geometry}
%\usepackage{lmodern} % Ensure lmodern is loaded for pdflatex
\usepackage{tfrupee} % Include tfrupee package

\setlength{\headheight}{1cm} % Set the height of the header box
\setlength{\headsep}{0mm}     % Set the distance between the header box and the top of the text

\usepackage{gvv-book}
\usepackage{gvv}
\usepackage{cite}
\usepackage{amsmath,amssymb,amsfonts,amsthm}
\usepackage{algorithmic}
\usepackage{graphicx}
\usepackage{textcomp}
\usepackage{xcolor}
\usepackage{txfonts}
\usepackage{listings}
\usepackage{enumitem}
\usepackage{mathtools}
\usepackage{gensymb}
\usepackage{comment}
\usepackage[breaklinks=true]{hyperref}
\usepackage{tkz-euclide} 
\usepackage{listings}
% \usepackage{gvv}                                        
\def\inputGnumericTable{}                                 
\usepackage[latin1]{inputenc}                                
\usepackage{color}                                            
\usepackage{array}                                            
\usepackage{longtable}                                       
\usepackage{calc}                                             
\usepackage{multirow}                                         
\usepackage{hhline}                                           
\usepackage{ifthen}                                           
\usepackage{lscape}
\usepackage{circuitikz}


\renewcommand{\thefigure}{\theenumi}
\renewcommand{\thetable}{\theenumi}
\setlength{\intextsep}{10pt} % Space between text and floats


\numberwithin{equation}{enumi}
\numberwithin{figure}{enumi}
\renewcommand{\thetable}{\theenumi}


% Marks the beginning of the document
\begin{document}
\bibliographystyle{IEEEtran}
\vspace{3cm}

\title{GATE 2010 MA(1-13)}
\author{EE24BTECH11030 - J.KEDARANANDA}
% \maketitle
% \newpage
% \bigskip
{\let\newpage\relax\maketitle}
\renewcommand{\thefigure}{\theenumi}
\renewcommand{\thetable}{\theenumi}
\begin{enumerate}
    \item Let $E$ and $F$ be any two events with $P(E \cup F') = 0.8, P(E) = 0.4$ and $P(E|F') = 0.3$. Then $P(F')$ is
    
    \begin{multicols}{4}
    \begin{enumerate}
        \item $\frac{3}{7}$
        \item $\frac{4}{7}$
        \item $\frac{3}{5}$
        \item $\frac{2}{5}$
    \end{enumerate}
    \end{multicols}

    \item Let $X$ have a binomial distribution with parameters $n$ and $p$, where $n$ is an integer greater than 1 and $0 < p < 1$. If $P(X = 0) = P(X = 1)$, then the value of $p$ is

    \begin{multicols}{4}
    \begin{enumerate}
        \item $\frac{1}{n-1}$
        \item $\frac{n}{n+1}$
        \item $\frac{1}{n+1}$
        \item $\frac{1}{1 + n^{n-1}}$
    \end{enumerate}
    \end{multicols}

    \item Let $u(x, y) = 2x(1 - y)$ for all real $x$ and $y$. Then a function $\varphi(x, y)$, so that $f(z) = u(x, y) + i\varphi(x, y)$ is analytic, is

    \begin{multicols}{4}
    \begin{enumerate}
        \item $(x - 1)^2 - y^2$
        \item $(x - 1)^2 + y^2$
        \item $(x - 1)^2 + y^2$
        \item $(x + 1)^2$
    \end{enumerate}
    \end{multicols}

    \item Let $f(z)$ be analytic on $D = \{z \in \mathbb{C} : |z - 1| < 1\}$ such that $f(0) = 1$. If $f(z) = f(z')$ for all $z \in D$, then which one of the following statements is NOT correct?

    \begin{multicols}{4}
    \begin{enumerate}
        \item $f(z) = [f(z)]^2 \text{ for all } z \in D$
        \item $f\left(\frac{z}{2}\right) = \frac{1}{2} f(z) \text{ for all } z \in D$
        \item $f(z) = [f(z)]^2 \text{ for all } z \in D$
        \item $f'(1) = 0$
    \end{enumerate}
    \end{multicols}

    \item The maximum number of linearly independent solutions of the differential equation $\frac{d^n y}{dx^n} = 0$ with the condition $y(0) = 1$ is

    \begin{multicols}{4}
    \begin{enumerate}
        \item 4
        \item 6
        \item 2
        \item 1
    \end{enumerate}
    \end{multicols}

    \item Which one of the following sets of functions is NOT orthogonal (with respect to the $L^2$-inner product) over the given interval?

    \begin{multicols}{4}
    \begin{enumerate}
        \item $\{\sin n\pi x : n \in \mathbb{N}\}, -\pi < x < \pi$
        \item $\{\cos n\pi x : n \in \mathbb{N}\}, -\pi < x < \pi$
        \item $\{\cos n\pi x : n \in \mathbb{N}\}, -1 < x < 1$
        \item $\{1 : n \in \mathbb{N}\}, -1 < x < 1$
    \end{enumerate}
    \end{multicols}

    \item If $f : [1, 2] \rightarrow \mathbb{R}$ is a non-negative Riemann-integrable function such that
    \[
    \int_1^2 \frac{f(x)}{x} dx = \int_1^1 f(x) dx = 0,
    \]
    then $a$ belongs to the interval

    \begin{multicols}{4}
    \begin{enumerate}
        \item $\left[0, \frac{1}{3}\right)$
        \item $\left[\frac{1}{2}, \frac{2}{3}\right)$
        \item $\left[\frac{2}{3}, 1\right)$
        \item $\left[1, \frac{4}{3}\right)$
    \end{enumerate}
    \end{multicols}
    \item The set $X = \mathbb{R}$ with the metric $d(x, y) = \frac{|x - y|}{1 + |x - y|}$ is
    \begin{multicols}{2}
    \begin{enumerate}
        \item bounded but not compact
        \item bounded but not complete
        \item complete but not bounded
        \item compact but not complete
    \end{enumerate}
    \end{multicols}

    \item Let $f(x, y) = 
    \begin{cases} 
        \frac{xy}{\left(x^2 + y^2\right)^{3/2}} \left[1 - \cos(x^2 + y^2)\right], & (x, y) \neq (0,0) \\ 
        k, & (x, y) = (0,0) 
    \end{cases}$

    Then the value of $k$ for which $f(x, y)$ is continuous at $(0,0)$ is
    \begin{multicols}{4}
    \begin{enumerate}
        \item 0
        \item $\frac{1}{2}$
        \item 1
        \item $\frac{3}{2}$
    \end{enumerate}
    \end{multicols}

    \item Let $A$ and $B$ be disjoint subsets of $\mathbb{R}$ and let $m^*$ denote the Lebesgue outer measure on $\mathbb{R}$.

    Consider the statements:
    \begin{itemize}
        \item[$P$:] $m^*(A \cup B) = m^*(A) + m^*(B)$
        \item[$Q$:] Both $A$ and $B$ are Lebesgue measurable
        \item[$R$:] One of $A$ and $B$ is Lebesgue measurable
    \end{itemize}
    
    Which one of the following is correct?
    \begin{multicols}{2}
    \begin{enumerate}
        \item If $P$ is true, then $Q$ is true
        \item If $P$ is NOT true, then $R$ is true
        \item If $R$ is true, then $P$ is NOT true
        \item If $R$ is true, then $P$ is true
    \end{enumerate}
    \end{multicols}

    \item Let $f : \mathbb{R} \rightarrow [0, \infty)$ be a Lebesgue measurable function and $E$ be a Lebesgue measurable subset of $\mathbb{R}$ such that $\int_E f \, dm = 0$, where $m$ is the Lebesgue measure on $\mathbb{R}$. Then
    \begin{multicols}{2}
    \begin{enumerate}
        \item $m(E) = 0$
        \item $\{x \in \mathbb{R} : f(x) = 0\} = E$
        \item $m(\{x \in E : f(x) \neq 0\}) = 0$
        \item $m(\{x \in E : f(x) = 0\}) = 0$
    \end{enumerate}
    \end{multicols}

    \item If the nullity of the matrix 
    $\begin{bmatrix} k & 1 & 2 \\ 1 & -1 & -2 \\ 1 & 1 & 4 \end{bmatrix}$ is 1, then the value of $k$ is
    \begin{multicols}{4}
    \begin{enumerate}
        \item -1
        \item 0
        \item 1
        \item 2
    \end{enumerate}
    \end{multicols}

    \item If a $3 \times 3$ real skew-symmetric matrix has an eigenvalue $2i$, then one of the remaining eigenvalues is
    \begin{multicols}{4}
    \begin{enumerate}
        \item $\frac{1}{2i}$
        \item $-\frac{1}{2i}$
        \item 0
        \item 1
    \end{enumerate}
    \end{multicols}
    
\end{enumerate}
\end{document}



