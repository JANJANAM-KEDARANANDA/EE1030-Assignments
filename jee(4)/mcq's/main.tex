\let\negmedspace\undefined
\let\negthickspace\undefined
\documentclass[journal]{IEEEtran}
\usepackage[a5paper, margin=10mm, onecolumn]{geometry}
%\usepackage{lmodern} % Ensure lmodern is loaded for pdflatex
\usepackage{tfrupee} % Include tfrupee package

\setlength{\headheight}{1cm} % Set the height of the header box
\setlength{\headsep}{0mm}     % Set the distance between the header box and the top of the text

\usepackage{gvv-book}
\usepackage{gvv}
\usepackage{cite}
\usepackage{amsmath,amssymb,amsfonts,amsthm}
\usepackage{algorithmic}
\usepackage{graphicx}
\usepackage{textcomp}
\usepackage{xcolor}
\usepackage{txfonts}
\usepackage{listings}
\usepackage{enumitem}
\usepackage{mathtools}
\usepackage{gensymb}
\usepackage{comment}
\usepackage[breaklinks=true]{hyperref}
\usepackage{tkz-euclide} 
\usepackage{listings}
% \usepackage{gvv}                                        
\def\inputGnumericTable{}                                 
\usepackage[latin1]{inputenc}                                
\usepackage{color}                                            
\usepackage{array}                                            
\usepackage{longtable}                                       
\usepackage{calc}                                             
\usepackage{multirow}                                         
\usepackage{hhline}                                           
\usepackage{ifthen}                                           
\usepackage{lscape}


\renewcommand{\thefigure}{\theenumi}
\renewcommand{\thetable}{\theenumi}
\setlength{\intextsep}{10pt} % Space between text and floats


\numberwithin{equation}{enumi}
\numberwithin{figure}{enumi}
\renewcommand{\thetable}{\theenumi}


% Marks the beginning of the document
\begin{document}
\bibliographystyle{IEEEtran}
\vspace{3cm}

\title{jee-main-maths-27-08-2021-shift-1}
\author{EE24BTECH11030 - J.KEDARANANDA}
% \maketitle
% \newpage
% \bigskip
{\let\newpage\relax\maketitle}
\renewcommand{\thefigure}{\theenumi}
\renewcommand{\thetable}{\theenumi}
\begin{enumerate}
    \item If $\alpha$, $\beta$ are the distinct roots of $x^2+ bx + c = 0$ , then $\lim_{x \to \beta}\frac{e^{2(x^2 + bx + c)} - 1 - 2(x^2 + bx + c)}{{(x - \beta)}^2}$ is equal to:  \\ 
    \begin{multicols}{4}
    \begin{enumerate}
        \item $b^2 + 4c$
        \item $2(b^2 + 4c)$
        \item $2(b^2 - 4c)$
        \item $b^2 - 4c$
    \end{enumerate}
    \end{multicols}
    \item When a certain biased die is rolled, a particular face occurs with probability $\frac{1}{6} - x$ and its opposite face occurs with probability $\frac{1}{6} + x$. All other faces occur with probability $\frac{1}{6}$ Note that opposite faces sum to 7 in any die. If 0 $<$ x $<$ $\frac{1}{6}$ and the probability of obtaining total sum = 7, when such a die is rolled twice, is $\frac{13}{96}$, then the value of x is:  \\
    \begin{multicols}{4}
    \begin{enumerate}
        \item $\frac{1}{16}$
        \item $\frac{1}{8}$
        \item $\frac{1}{9}$
        \item $\frac{1}{12}$
    \end{enumerate} 
    \end{multicols}
    \item If $x^2+ 9y^2 - 4x + 3 = 0$, x, y $\in$ R , then x and y respectively lie in the intervals:  \\
    \begin{multicols}{2}
    \begin{enumerate}
        \item $\left[\frac{-1}{3},\frac{1}{3}\right]$ and $\left[\frac{-1}{3},\frac{1}{3}\right]$\\
        \item $\left[\frac{-1}{3},\frac{1}{3}\right]$ and $\left[1,3\right]$
        \item $\left[1,3\right]$ and $\left[1,3\right]$\\
        \item $\left[1,3\right]$ and $\left[\frac{-1}{3},\frac{1}{3}\right]$
    \end{enumerate}
    \end{multicols}
    \item $\int_{6}^{16} \frac{\ln x^2}{\ln{x^2} + \ln{(x^2 - 44x + 484)}} \, dx$ is equal to: \\
    \begin{multicols}{4}
    \begin{enumerate}
        \item 6
        \item 8
        \item 5
        \item 10
    \end{enumerate} 
    \end{multicols}
    \item A wire of length 20 m is to be cut into two pieces.One of the pieces is to be made into a square and the other into a regular hexagon. Then the length of the side (in meters) of the hexagon, so that the combined area of the square and the hexagon is minimum, is:  \\
    \begin{multicols}{4}
    \begin{enumerate}
        \item $\frac{5}{2 + \sqrt{3}}$
        \item $\frac{10}{2 + 3\sqrt{3}}$
        \item $\frac{5}{3 + \sqrt{3}}$
        \item $\frac{10}{3 + 2\sqrt{3}}$
    \end{enumerate} 
    \end{multicols}
\end{enumerate}
\end{document}
