\let\negmedspace\undefined
\let\negthickspace\undefined
\documentclass[journal]{IEEEtran}
\usepackage[a5paper, margin=10mm, onecolumn]{geometry}
%\usepackage{lmodern} % Ensure lmodern is loaded for pdflatex
\usepackage{tfrupee} % Include tfrupee package

\setlength{\headheight}{1cm} % Set the height of the header box
\setlength{\headsep}{0mm}     % Set the distance between the header box and the top of the text

\usepackage{gvv-book}
\usepackage{gvv}
\usepackage{cite}
\usepackage{amsmath,amssymb,amsfonts,amsthm}
\usepackage{algorithmic}
\usepackage{graphicx}
\usepackage{textcomp}
\usepackage{xcolor}
\usepackage{txfonts}
\usepackage{listings}
\usepackage{enumitem}
\usepackage{mathtools}
\usepackage{gensymb}
\usepackage{comment}
\usepackage[breaklinks=true]{hyperref}
\usepackage{tkz-euclide} 
\usepackage{listings}
% \usepackage{gvv}                                        
\def\inputGnumericTable{}                                 
\usepackage[latin1]{inputenc}                                
\usepackage{color}                                            
\usepackage{array}                                            
\usepackage{longtable}                                       
\usepackage{calc}                                             
\usepackage{multirow}                                         
\usepackage{hhline}                                           
\usepackage{ifthen}                                           
\usepackage{lscape}


\renewcommand{\thefigure}{\theenumi}
\renewcommand{\thetable}{\theenumi}
\setlength{\intextsep}{10pt} % Space between text and floats


\numberwithin{equation}{enumi}
\numberwithin{figure}{enumi}
\renewcommand{\thetable}{\theenumi}


% Marks the beginning of the document
\begin{document}
\bibliographystyle{IEEEtran}
\vspace{3cm}

\title{jee-main-maths-27-08-2021-shift-1}
\author{EE24BTECH11030 - J.KEDARANANDA}
% \maketitle
% \newpage
% \bigskip
{\let\newpage\relax\maketitle}
\renewcommand{\thefigure}{\theenumi}
\renewcommand{\thetable}{\theenumi}
\begin{enumerate}
    \item Let $\overset{\rightarrow}{a} = \hat{i} + 5\hat{j} + \alpha \hat{k}$ , $\overset{\rightarrow}{b} = \hat{i} + 3\hat{j} + \beta \hat{k}$ and $\overset{\rightarrow}{c} = -1\hat{i} + 2\hat{j} - 3\hat{k}$ be three vectors such that $|\overset{\rightarrow}{b} \times \overset{\rightarrow}{c}|$ = $5\sqrt{3}$ and
    $\overset{\rightarrow}{a}$ is perpendicular to $\overset{\rightarrow}{b}$. Then the greatest among the values of ${|\overset{\rightarrow}{a}|}^2$ is \underline{\hspace{1cm}}.\\\\
    \item The number of distinct real roots of the equation $3x^4 + 4x^3 - 12x^2 + 4 = 0$ is  \underline{\hspace{1cm}}. \\\\
    \item Let the equation $x^2 + y^2 + px + (1 - p)y + 5 = 0$ represent circles of varying radius r $\in$ (0, 5]. Then the number of elements in the set S = \{q : q = $p^2$ and q is an integer\} is \underline{\hspace{1cm}}. \\\\
    \item If $A = \{ x \in R : |x - 2| > 1 \}$, $B = \{ x \in R : \sqrt{x^2 - 3} > 1 \}$, $C = \{ x \in  R : |x - 4| \geq 2 \}$, and $Z$ is the set of all integers, then the number of subsets of the set $(A \cap B \cap C)^{c} \cap Z$ is \underline{\hspace{1cm}}. \\\\
    \item If $\int\frac{dx}{(x^2 + x +1)^{2}}$ = $a\tan^{-1}{\left(\frac{2x + 1}{\sqrt{3}}\right)} + b\left(\frac{2x + 1}{x^2 + x + 1}\right) + C$, x $>$ 0 where C is the constant of integration, then the value of $9(\sqrt{3}a + b)$ is equal to \underline{\hspace{1cm}}.\\\\
    \item If the system of linear equations\\ $2x + y - z = 3$\\ $x - y - z = \alpha$\\ $3x + 3y + \beta z = 3$\\ has infinitely many solution, then $\alpha + \beta - \alpha\beta$ is equal to \underline{\hspace{1cm}}. \\\\
    \item Let n be an odd natural number such that the variance of 1, 2, 3, 4, $\cdots$, n is 14. Then n is equal to \underline{\hspace{1cm}}. \\\\
    \item If the minimum area of the triangle formed by a tangent to the ellipse $\frac{x^{2}}{b^{2}} + \frac{y^{2}}{4a^{2}} = 1$ and the co-ordinate axis is kab, then k is equal to \underline{\hspace{1cm}}. \\\\
    \item A number is called a palindrome if it reads the same backward as well as forward. For example 285582 is a six digit palindrome. The number of six digit palindromes, which are divisible by 55, is \underline{\hspace{1cm}}. \\\\
    \item If $y^{\frac{1}{4}} + y^{\frac{-1}{4}} = 2x$, and $(x^{2} - 1)\frac{d^{2}y}{dx^{2}} + \alpha x\frac{dy}{dx} + \beta y= 0$, then $|\alpha - \beta|$ is equal to \underline{\hspace{1cm}}. \\\\
\end{enumerate}
\end{document}

