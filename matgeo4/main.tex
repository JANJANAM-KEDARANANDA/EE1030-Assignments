\let\negmedspace\undefined
\let\negthickspace\undefined
\documentclass[journal]{IEEEtran}
\usepackage[a5paper, margin=10mm, onecolumn]{geometry}
%\usepackage{lmodern} % Ensure lmodern is loaded for pdflatex
\usepackage{tfrupee} % Include tfrupee package

\setlength{\headheight}{1cm} % Set the height of the header box
\setlength{\headsep}{0mm}     % Set the distance between the header box and the top of the text
\usepackage{gvv-book}
\usepackage{gvv}
\usepackage{cite}
\usepackage{amsmath,amssymb,amsfonts,amsthm}
\usepackage{algorithmic}
\usepackage{graphicx}
\usepackage{textcomp}
\usepackage{xcolor}
\usepackage{txfonts}
\usepackage{listings}
\usepackage{enumitem}
\usepackage{mathtools}
\usepackage{gensymb}
\usepackage{comment}
\usepackage[breaklinks=true]{hyperref}
\usepackage{tkz-euclide} 
\usepackage{listings}
% \usepackage{gvv}                                        
\def\inputGnumericTable{}                                 
\usepackage[latin1]{inputenc}                                
\usepackage{color}                                            
\usepackage{array}                                            
\usepackage{longtable}                                       
\usepackage{calc}                                             
\usepackage{multirow}                                         
\usepackage{hhline}                                           
\usepackage{ifthen}                                           
\usepackage{lscape}
\begin{document}

\bibliographystyle{IEEEtran}
\vspace{3cm}

\title{1.1.9.17}
\author{EE24BTECH11030 - J.KEDARANANDA}
% \maketitle
% \newpage
% \bigskip
{\let\newpage\relax\maketitle}

\renewcommand{\thefigure}{\theenumi}
\renewcommand{\thetable}{\theenumi}
\setlength{\intextsep}{10pt} % Space between text and floats


\numberwithin{equation}{enumi}
\numberwithin{figure}{enumi}
\renewcommand{\thetable}{\theenumi}


\textbf{Question}:\\
Write the coordinates of the point $\vec{P}$ on the x-axis which is equidistant from the points $\vec{A}\myvec{-2,0}$ $\vec{B}\myvec{6,0}$.\hfill{\brak{10,2019}}\\
\\ \textbf{Solution: }\\
    \begin{table}[h!]    
      \centering
      \begin{tabular}[12pt]{ |c| c |c|}
    \hline
    \textbf{Variable} & \textbf{Description} & \textbf{Formula}\\ 
    \hline
    $\myvec{x_1\\y_1}$ & x,y coordinate of P respectively & $\frac{k(\vec{B})+(\vec{A})}{k+1}$ \\
    \hline 
    $\myvec{x_2\\y_2}$ & x,y coordinate of Q respectively & $\frac{k(\vec{B})+(\vec{A})}{k+1}$ \\
    \hline  
    $\myvec{2\\-2}$ & x,y coordinate of A respectively & \\
    \hline
    $\myvec{-7\\4}$ & x,y coordinate of B respectively & \\
    \hline  
    \end{tabular}


      \caption{}
    \end{table}\\
If \textbf{P} is equidistant from the points \textbf{A} and \textbf{B}
    \begin{align}
        \abs{\abs{\boldsymbol{P}-\boldsymbol{A}}} = \abs{\abs{\boldsymbol{P}-\boldsymbol{B}}} \label{eq1.1.8.28.1}
    \end{align}
     \begin{align}
        \abs{\abs{\boldsymbol{P}-\boldsymbol{A}}}^2 = \abs{\abs{\boldsymbol{P}-\boldsymbol{B}}}^2\label{eq1.1.8.28.2}
    \end{align}
    \begin{align}
    	\abs{\abs{\boldsymbol{P}}}^2 - 2\boldsymbol{P}^\top \boldsymbol{A} + \abs{\abs{\boldsymbol{A}}}^2 = \abs{\abs{\boldsymbol{P}}}^2 - 2\boldsymbol{P}^\top\boldsymbol{B} + \abs{\abs{\boldsymbol{B}}}^2\label{eq1.1.8.28.3}
    \end{align}
By simplifying further,
    \begin{align}
    	\brak{\boldsymbol{A}-\boldsymbol{B}}^\top \vec{P} = \frac{\abs{\abs{\boldsymbol{A}}}^2-\abs{\abs{\boldsymbol{B}}}^2}{2} \\
    	\myvec{-8 \\ 0}^\top \vec{P} = \frac{\abs{\abs{\myvec{-2 \\ 0}}}^2-\abs{\abs{\myvec{6 \\ 0}}}^2}{2} = -16\label{eq1.1.8.28.4}
    \end{align}
Comparing with $n^\top x = c$
    \begin{align}
    	\boldsymbol{n} = \myvec{-8 \\ 0}\\ \label{eq1.1.8.28.5}
	    \boldsymbol{c} = -16\\
     -8x+0y=-16\\
     x=2,y=0\\
     \vec{P}=\myvec{2 \\ 0}
    \end{align}
    \begin{figure}[h]
        \centering
       \includegraphics[width=0.7\linewidth]{figs/fig1.png}
       \caption{}
       \label{graph}
    \end{figure}



\end{document}  







