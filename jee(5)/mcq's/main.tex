\let\negmedspace\undefined
\let\negthickspace\undefined
\documentclass[journal]{IEEEtran}
\usepackage[a5paper, margin=10mm, onecolumn]{geometry}
%\usepackage{lmodern} % Ensure lmodern is loaded for pdflatex
\usepackage{tfrupee} % Include tfrupee package

\setlength{\headheight}{1cm} % Set the height of the header box
\setlength{\headsep}{0mm}     % Set the distance between the header box and the top of the text

\usepackage{gvv-book}
\usepackage{gvv}
\usepackage{cite}
\usepackage{amsmath,amssymb,amsfonts,amsthm}
\usepackage{algorithmic}
\usepackage{graphicx}
\usepackage{textcomp}
\usepackage{xcolor}
\usepackage{txfonts}
\usepackage{listings}
\usepackage{enumitem}
\usepackage{mathtools}
\usepackage{gensymb}
\usepackage{comment}
\usepackage[breaklinks=true]{hyperref}
\usepackage{tkz-euclide} 
\usepackage{listings}
% \usepackage{gvv}                                        
\def\inputGnumericTable{}                                 
\usepackage[latin1]{inputenc}                                
\usepackage{color}                                            
\usepackage{array}                                            
\usepackage{longtable}                                       
\usepackage{calc}                                             
\usepackage{multirow}                                         
\usepackage{hhline}                                           
\usepackage{ifthen}                                           
\usepackage{lscape}


\renewcommand{\thefigure}{\theenumi}
\renewcommand{\thetable}{\theenumi}
\setlength{\intextsep}{10pt} % Space between text and floats


\numberwithin{equation}{enumi}
\numberwithin{figure}{enumi}
\renewcommand{\thetable}{\theenumi}


% Marks the beginning of the document
\begin{document}
\bibliographystyle{IEEEtran}
\vspace{3cm}

\title{jee-main-maths-25-06-2022-shift-2}
\author{EE24BTECH11030 - J.KEDARANANDA}
% \maketitle
% \newpage
% \bigskip
{\let\newpage\relax\maketitle}
\renewcommand{\thefigure}{\theenumi}
\renewcommand{\thetable}{\theenumi}
\begin{enumerate}
    \item A biased die is marked with numbers 2, 4, 8, 16, 32, 32 on its faces and the probability of getting a face with mark n is $\frac{1}{n}$. If the die is thrown thrice, then the probability, that the sum of the numbers obtained is 48, is :  \\ 
    \begin{multicols}{4}
    \begin{enumerate}
        \item $\frac{7}{2^{11}}$
        \item $\frac{7}{2^{12}}$
        \item $\frac{3}{2^{10}}$
        \item $\frac{13}{2^{12}}$
    \end{enumerate}
    \end{multicols}
    \bigskip
    \item The negation of the Boolean expression $((\sim q) \land p) \implies ((\sim p) \lor q)$ is logically equivalent to :  \\
    \begin{multicols}{4}
    \begin{enumerate}
        \item $p \implies q$
        \item $q \implies p$
        \item $\sim (p \implies q)$
        \item $\sim (q \implies p)$
    \end{enumerate} 
    \end{multicols}
    \bigskip
    \item If the line $y = 4 + kx$, k $>$ 0, is the tangent to the parabola $y = x - x^2$ at the point $\vec{P}$ and $\vec{V}$ is the vertex of the parabola, then the slope of the line through $\vec{P}$ and $\vec{V}$ is :  \\
    \begin{multicols}{4}
    \begin{enumerate}
        \item $\frac{3}{2}$
        \item $\frac{26}{9}$
        \item $\frac{5}{2}$
        \item $\frac{23}{6}$
    \end{enumerate}
    \end{multicols}
    \bigskip
    \item The value of $tan^{-1}{\left(\frac{\cos{\frac{15\pi}{4}} - 1}{\sin{\frac{\pi}{4}}}\right)}$ is equal to: \\
    \begin{multicols}{4}
    \begin{enumerate}
        \item $\frac{-\pi}{4}$
        \item $\frac{-\pi}{8}$
        \item $\frac{-5\pi}{12}$
        \item $\frac{-4\pi}{9}$
    \end{enumerate} 
    \end{multicols}
    \bigskip
    \item  The line $y = x + 1$ meets the ellipse $\frac{x^2}{4} + \frac{y^2}{2} = 1$ at two points $\vec{P}$ and $\vec{Q}$. If r is the radius of the circle with PQ as diameter then $(3r)^2$ is equal to : \\
    \begin{multicols}{4}
    \begin{enumerate}
        \item 20
        \item 12
        \item 11
        \item 8
    \end{enumerate} 
    \end{multicols}
    \bigskip
\end{enumerate}
\end{document}

