\let\negmedspace\undefined
\let\negthickspace\undefined
\documentclass[journal]{IEEEtran}
\usepackage[a5paper, margin=10mm, onecolumn]{geometry}
%\usepackage{lmodern} % Ensure lmodern is loaded for pdflatex
\usepackage{tfrupee} % Include tfrupee package

\setlength{\headheight}{1cm} % Set the height of the header box
\setlength{\headsep}{0mm}     % Set the distance between the header box and the top of the text

\usepackage{gvv-book}
\usepackage{gvv}
\usepackage{cite}
\usepackage{amsmath,amssymb,amsfonts,amsthm}
\usepackage{algorithmic}
\usepackage{graphicx}
\usepackage{textcomp}
\usepackage{xcolor}
\usepackage{txfonts}
\usepackage{listings}
\usepackage{enumitem}
\usepackage{mathtools}
\usepackage{gensymb}
\usepackage{comment}
\usepackage[breaklinks=true]{hyperref}
\usepackage{tkz-euclide} 
\usepackage{listings}
% \usepackage{gvv}                                        
\def\inputGnumericTable{}                                 
\usepackage[latin1]{inputenc}                                
\usepackage{color}                                            
\usepackage{array}                                            
\usepackage{longtable}                                       
\usepackage{calc}                                             
\usepackage{multirow}                                         
\usepackage{hhline}                                           
\usepackage{ifthen}                                           
\usepackage{lscape}


\renewcommand{\thefigure}{\theenumi}
\renewcommand{\thetable}{\theenumi}
\setlength{\intextsep}{10pt} % Space between text and floats


\numberwithin{equation}{enumi}
\numberwithin{figure}{enumi}
\renewcommand{\thetable}{\theenumi}


% Marks the beginning of the document
\begin{document}
\bibliographystyle{IEEEtran}
\vspace{3cm}

\title{jee-main-maths-25-06-2022-shift-2}
\author{EE24BTECH11030 - J.KEDARANANDA}
% \maketitle
% \newpage
% \bigskip
{\let\newpage\relax\maketitle}
\renewcommand{\thefigure}{\theenumi}
\renewcommand{\thetable}{\theenumi}
\begin{enumerate}
    \item Let A = $\begin{pmatrix} 2 & -2 \\ 1 & -1 \end{pmatrix}$ and  B = $\begin{pmatrix} -1 & 2 \\ -1 & 2 \end{pmatrix}$\\
    Then the number of elements in the set {(n, m) : n, m $\in$ { 1, 2,$\cdots$, 10} and $nA^n + mB^m = I$} is \underline{\hspace{1cm}}.
    \bigskip
    
    \item Let $f(x) = [2x^2 + 1]$ and 
    $g(x) = \begin{cases} 
    2x - 3 & , x < 0 \\ 
    2x + 3 & , x \geq 0 
    \end{cases}$, where  [t] is the greatest integer $\leq t$. Then, in the open interval $(-1, 1)$, the number of points where $f(g(x))$ is discontinuous is equal to \underline{\hspace{1cm}}.
    \bigskip
    
    \item The value of b $>$ 3 for which $12\int_{3}^{b}\frac{1}{(x^2 - 1)(x^2 - 4)} \,dx = \ln{(\frac{49}{40})}$, is equal to \underline{\hspace{1cm}}.
    \bigskip
    
    \item If the sum of the co-efficients of all the positive even powers of x in the binomial expansion of $\left(2x^3 + \frac{3}{x}\right)^{10}$ is $5^{10} - {\beta}3^9$ then $\beta$ equal to \underline{\hspace{1cm}}
    \bigskip
    
    \item If the mean deviation about the mean of the numbers 1, 2, 3, $\cdots$, n, where n is odd, is $\frac{5(n+1)}{n}$, then n is equal to \underline{\hspace{1cm}}.
    \bigskip
    
    \item $\overset{\rightarrow}{b} = \hat{i} + \hat{j} + \lambda\hat{k}$ , $\lambda \in R$ . If $\overset{\rightarrow}{b}$ is a vector such that $\overset{\rightarrow}{a} \times \overset{\rightarrow}{b}$ = $13\hat{i} - 1\hat{j} - 4\lambda\hat{k}$ and $\overset{\rightarrow}{a} \cdot \overset{\rightarrow}{b} + 21 = 0$, then $(\overset{\rightarrow}{b} - \overset{\rightarrow}{a}) \cdot (\hat{k} - \hat{j}) + (\overset{\rightarrow}{b} + \overset{\rightarrow}{a}) \cdot (\hat{i} - \hat{k})$ \underline{\hspace{1cm}}. 
    \bigskip
    
    \item The total number of three-digit numbers, with one digit repeated exactly two times, is \underline{\hspace{1cm}}. 
    \bigskip
    
    \item Let f(x) = $|(x - 1)(x^2 - 2x - 3)| + x - 3$, x $\in$ R. If m and M are, respectively the number of points of local minimum and local maximum of f in the interval (0, 4), then m + M is equal to \underline{\hspace{1cm}}. \\\\
    \bigskip
    
    \item Let the eccentricity of the hyperbola $\frac{x^2}{a^2} - \frac{y^2}{b^2} = 1$ be $\frac{5}{4}$. If the equation of the normal at the point $(\frac{8}{\sqrt{5}},\frac{12}{5})$ on the hyperbola is $8\sqrt{5}x + \beta y = \lambda$, then $\lambda - \beta$ is equal to \underline{\hspace{1cm}}. 
    \bigskip
    
    \item Let $l_1$ be the line in xy-plane with x and y intercepts $\frac{1}{8}$ and $\frac{1}{4\sqrt{2}}$ respectively and $l_2$ be the line in zx-plane with x and z intercepts $\frac{-1}{8}$ and $\frac{-1}{6\sqrt{3}}$ respectively. If d is the shortest distance between the line $l_1$ and $l_2$, then $d^{-2}$ is equal to \underline{\hspace{1cm}}. \\\\
\end{enumerate}
\end{document}


