\let\negmedspace\undefined
\let\negthickspace\undefined
\documentclass[journal]{IEEEtran}
\usepackage[a5paper, margin=10mm, onecolumn]{geometry}
%\usepackage{lmodern} % Ensure lmodern is loaded for pdflatex
\usepackage{tfrupee} % Include tfrupee package

\setlength{\headheight}{1cm} % Set the height of the header box
\setlength{\headsep}{0mm}     % Set the distance between the header box and the top of the text

\usepackage{gvv-book}
\usepackage{gvv}
\usepackage{cite}
\usepackage{amsmath,amssymb,amsfonts,amsthm}
\usepackage{algorithmic}
\usepackage{graphicx}
\usepackage{textcomp}
\usepackage{xcolor}
\usepackage{txfonts}
\usepackage{listings}
\usepackage{enumitem}
\usepackage{mathtools}
\usepackage{gensymb}
\usepackage{comment}
\usepackage[breaklinks=true]{hyperref}
\usepackage{tkz-euclide} 
\usepackage{listings}
% \usepackage{gvv}                                        
\def\inputGnumericTable{}                                 
\usepackage[latin1]{inputenc}                                
\usepackage{color}                                            
\usepackage{array}                                            
\usepackage{longtable}                                       
\usepackage{calc}                                             
\usepackage{multirow}                                         
\usepackage{hhline}                                           
\usepackage{ifthen}                                           
\usepackage{lscape}


\renewcommand{\thefigure}{\theenumi}
\renewcommand{\thetable}{\theenumi}
\setlength{\intextsep}{10pt} % Space between text and floats


\numberwithin{equation}{enumi}
\numberwithin{figure}{enumi}
\renewcommand{\thetable}{\theenumi}


% Marks the beginning of the document
\begin{document}
\bibliographystyle{IEEEtran}
\vspace{3cm}

\title{jee-main-maths-13-04-2023-shift-2}
\author{EE24BTECH11030 - J.KEDARANANDA}
% \maketitle
% \newpage
% \bigskip
{\let\newpage\relax\maketitle}
\renewcommand{\thefigure}{\theenumi}
\renewcommand{\thetable}{\theenumi}
\begin{enumerate}
    \item The random variable X follows binomial distribution B (n, p), for which the difference of the mean and the variance is 1. If 2$\mathbb{P}$(x = 2) = 3$\mathbb{P}$(x = 1), then \\$n^2\mathbb{P}$(X $>$ 1) is equal to  \\ 
    \begin{multicols}{4}
    \begin{enumerate}
        \item 16
        \item 11
        \item 12
        \item 15
    \end{enumerate}
    \end{multicols}
    \bigskip
    \item Let the centre of a circle C be ($\alpha$ , $\beta$) and its radius r $<$ 8. Let 3x + 4y = 24 and\\ 3x - 4y = 32 be two tangents and 4x + 3y = 1 be a normal to C. Then ($\alpha$ - $\beta$ + r) is equal to  \\
    \begin{multicols}{4}
    \begin{enumerate}
        \item 5
        \item 6
        \item 7
        \item 9
    \end{enumerate} 
    \end{multicols}
    \bigskip
    \item Let N be the foot of perpendicular from the point P (1, -2, 3) on the line passing through the points (4,5,8) and (1,-7,5). Then the distance of N from the plane\\ 2x - 2y + z + 5 = 0 is  \\
    \begin{multicols}{4}
    \begin{enumerate}
        \item 6
        \item 7
        \item 9
        \item 8
    \end{enumerate}
    \end{multicols}
    \bigskip
    \item All words, with or without meaning, are made using all the letters of the word MONDAY. These words are written as in a dictionary with serial numbers. The serial number of the word MONDAY is \\
    \begin{multicols}{4}
    \begin{enumerate}
        \item 328
        \item 327
        \item 324
        \item 326
    \end{enumerate} 
    \end{multicols}
    \bigskip
    \item  Let ($\alpha$ , $\beta$) be the centroid of the triangle formed by the lines 15x - y = 82, \\6x - 5y = -4 and 9x + 4y = 17. Then $\alpha$ + 2$\beta$ and 2$\alpha$ - $\beta$ are the roots of the equation \\
    \begin{multicols}{4}
    \begin{enumerate}
        \item $x^2 - 13x + 42 = 0$
        \item $x^2 - 10x + 25 = 0$
        \item $x^2 - 7x + 12 = 0$
        \item $x^2 - 14x + 48 = 0$
    \end{enumerate} 
    \end{multicols}
    \bigskip
\end{enumerate}
\end{document}


