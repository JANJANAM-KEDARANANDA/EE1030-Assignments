\let\negmedspace\undefined
\let\negthickspace\undefined
\documentclass[journal]{IEEEtran}
\usepackage[a5paper, margin=10mm, onecolumn]{geometry}
%\usepackage{lmodern} % Ensure lmodern is loaded for pdflatex
\usepackage{tfrupee} % Include tfrupee package

\setlength{\headheight}{1cm} % Set the height of the header box
\setlength{\headsep}{0mm}     % Set the distance between the header box and the top of the text

\usepackage{gvv-book}
\usepackage{gvv}
\usepackage{cite}
\usepackage{amsmath,amssymb,amsfonts,amsthm}
\usepackage{algorithmic}
\usepackage{graphicx}
\usepackage{textcomp}
\usepackage{xcolor}
\usepackage{txfonts}
\usepackage{listings}
\usepackage{enumitem}
\usepackage{mathtools}
\usepackage{gensymb}
\usepackage{comment}
\usepackage[breaklinks=true]{hyperref}
\usepackage{tkz-euclide} 
\usepackage{listings}
% \usepackage{gvv}                                        
\def\inputGnumericTable{}                                 
\usepackage[latin1]{inputenc}                                
\usepackage{color}                                            
\usepackage{array}                                            
\usepackage{longtable}                                       
\usepackage{calc}                                             
\usepackage{multirow}                                         
\usepackage{hhline}                                           
\usepackage{ifthen}                                           
\usepackage{lscape}
\usepackage{circuitikz}


\renewcommand{\thefigure}{\theenumi}
\renewcommand{\thetable}{\theenumi}
\setlength{\intextsep}{10pt} % Space between text and floats


\numberwithin{equation}{enumi}
\numberwithin{figure}{enumi}
\renewcommand{\thetable}{\theenumi}


% Marks the beginning of the document
\begin{document}
\bibliographystyle{IEEEtran}
\vspace{3cm}

\title{GATE 2016 XE(53-65)}
\author{EE24BTECH11030 - J.KEDARANANDA}
% \maketitle
% \newpage
% \bigskip
{\let\newpage\relax\maketitle}
\renewcommand{\thefigure}{\theenumi}
\renewcommand{\thetable}{\theenumi}
\begin{enumerate}
    \item In a diffraction experiment, monochromatic X-rays of wavelength $1.54 \, \text{\AA}$ are used to examine a material with a BCC structure. If the lattice parameter is $4.1 \, \text{\AA}$, the angular position $\theta$ of the first diffraction peak is \underline{\hspace{1cm}} degrees.
    
    \bigskip
    
    \item The yield strength of a ferritic steel increases from $120 \, \text{MPa}$ to $150 \, \text{MPa}$ when the grain size is decreased from $256 \, \mu\text{m}$ to $64 \, \mu\text{m}$. When the grain size is further reduced to $16 \, \mu\text{m}$, the expected yield strength is \underline{\hspace{1cm}} MPa.
    
    \bigskip

    \item A direct bandgap semiconductor has a bandgap of $1.8 \, \text{eV}$. The threshold value of the wavelength {\textbf{BELOW}} which this material will absorb radiation is \underline{\hspace{1cm}} $\text{\AA}$.
    
    (Given: Planck's constant, $h = 6.626 \times 10^{-34} \, \text{J s}$, the charge of an electron, $e = 1.6 \times 10^{-19} \, \text{C}$, and speed of light, $c = 3 \times 10^8 \, \text{m s}^{-1}$)
    
    \bigskip

    \item A half cell consisting of pure Ni immersed in an aqueous solution containing $\text{Ni}^{2+}$ ions of unknown concentration, is galvanically coupled with another half cell consisting of pure Cd immersed in a $1 \, \text{M}$ aqueous solution of $\text{Cd}^{2+}$ ions. The temperature is $25 \, ^\circ \text{C}$ and pressure is $1 \, \text{atm}$. The standard electrode reduction potentials of Ni and Cd are $-0.250 \, \text{V}$ and $-0.403 \, \text{V}$, respectively. The voltage of the cell is found to be zero. The concentration of $\text{Ni}^{2+}$ in the solution is \underline{\hspace{1cm}} $\times 10^{-6} \, \text{M}$.
    
    (Given: Universal gas constant, $R = 8.31 \, \text{J mol}^{-1} \text{K}^{-1}$, Faraday's constant, $F = 96500 \, \text{C mol}^{-1}$)
    
    \bigskip

    \item Match the type of magnetism given in Group 1 with the material given in Group 2:
    
    \[
    \begin{array}{|c|c|}
    \hline
    \text{Group 1} & \text{Group 2} \\
    \hline
    P: \text{Ferromagnetic} & 1: \text{Nickel oxide} \\
    Q: \text{Ferrimagnetic} & 2: \text{Sodium} \\
    R: \text{Antiferromagnetic} & 3: \text{Magnetite} \\
    S: \text{Paramagnetic} & 4: \text{Cobalt} \\
    \hline
    \end{array}
    \]
    
    \begin{enumerate}
        \item $(A) \quad P: 4, \, Q: 3, \, R: 1, \, S: 2$
        \item $(B) \quad P: 4, \, Q: 1, \, R: 3, \, S: 2$
        \item $(C) \quad P: 1, \, Q: 2, \, R: 4, \, S: 3$
        \item $(D) \quad P: 3, \, Q: 2, \, R: 1, \, S: 4$
    \end{enumerate}
    
    \bigskip

    \item Gallium is to be diffused into pure silicon wafer such that its concentration at a depth of $10^{-3} \, \text{cm}$ will be one half the surface concentration. Given that the diffusion coefficient $(D)$ of gallium in silicon at $1355 \, ^\circ \text{C}$ is $6 \times 10^{-11} \, \text{cm}^2 \, \text{s}^{-1}$, the time the silicon wafer should be heated in contact with gallium vapor at $1355 \, ^\circ \text{C}$ is \underline{\hspace{1cm}} s.
    
    (Given: $\text{erf}(0.5) \approx 0.5$)
    \bigskip
    \item A batch of spherical titania nanoparticles, uniform in size, has a specific surface area of $125 \, \text{m}^2 \, \text{g}^{-1}$. If the density of titania is $4.23 \, \text{g} \, \text{cm}^{-3}$, the diameter of the particles is \underline{\hspace{1cm}} nm.
    
    \bigskip

    \item Given the probability distribution function
    \[
    f(x) = 
    \begin{cases} 
      0.25x & \text{for } 1 \leq x \leq 3 \\
      0 & \text{otherwise}
    \end{cases}
    \]
    The probability that the random variable $x$ takes a value between $1$ and $\sqrt{5}$ is \underline{\hspace{1cm}}.
    
    \bigskip

    \item In the vulcanization of $50 \, \text{g}$ of natural rubber, $10 \, \text{g}$ of sulfur is added. Assuming the mer to S ratio is $1:1$, the maximum percentage of cross-linked sites that could be connected is \underline{\hspace{1cm}} \%. (Given: atomic weight of S is $32 \, \text{amu}$ and molecular weight of a mer of natural rubber is $68 \, \text{amu}$)
    
    \bigskip

    \item Match the heat treatment process of steels given in Group 1 with the microstructural feature given in Group 2:
    
    \[
    \begin{array}{|c|c|}
    \hline
    \text{Group 1} & \text{Group 2} \\
    \hline
    P: \text{Quenching} & 1: \text{Bainite} \\
    Q: \text{Normalizing} & 2: \text{Martensite} \\
    R: \text{Tempering} & 3: \text{Pearlite} \\
    S: \text{Austempering} & 4: \text{Iron carbide precipitates} \\
    & 5: \text{Intermetallic precipitates} \\
    \hline
    \end{array}
    \]
    
    \begin{enumerate}
        \item $(A) \quad P: 2, \, Q: 3, \, R: 4, \, S: 1$
        \item $(B) \quad P: 3, \, Q: 4, \, R: 5, \, S: 1$
        \item $(C) \quad P: 4, \, Q: 1, \, R: 5, \, S: 3$
        \item $(D) \quad P: 2, \, Q: 5, \, R: 4, \, S: 3$
    \end{enumerate}
    
    \bigskip

    \item In the photoelectric effect, electrons are ejected
    \begin{enumerate}
        \item at all wavelengths, as long as the intensity of the incident radiation is above a threshold value.
        \item at all wavelengths, as long as the intensity of the incident radiation is below a threshold value.
        \item at all intensities, as long as the wavelength of the incident radiation is below a threshold value.
        \item at all intensities, as long as the wavelength of the incident radiation is above a threshold value.
    \end{enumerate}
    
    \bigskip

    \item The angle between $[110]$ and $[111]$ directions in the cubic system is \underline{\hspace{1cm}} degrees.
    \bigskip
    \item A single degree of freedom vibrating system has mass of 5 kg, stiffness of 500 N/m and damping
    coefficient of 100 N-s/m. To make the system critically damped 
    \begin{enumerate}
        \item only the mass is to be increased by 1.2 times. 
        \item only the stiffness is to be reduced to half. 
        \item only the damping coefficient is to be doubled. 
        \item no change in any of the system parameters is required.
    \end{enumerate}
\end{enumerate}
\end{document}






